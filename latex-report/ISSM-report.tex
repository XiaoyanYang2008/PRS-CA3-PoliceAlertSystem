% Template for ICIP-2019 paper; to be used with:
%          spconf.sty  - ICASSP/ICIP LaTeX style file, and
%          IEEEbib.bst - IEEE bibliography style file.
% --------------------------------------------------------------------------
\documentclass{article}
\usepackage{spconf,amsmath,graphicx}

% Example definitions.
% --------------------
\def\x{{\mathbf x}}
\def\L{{\cal L}}

% Title.
% ------
\title{ISSM report, Police Alert System}
%
% Single address.
% ---------------
\name{NG SIEW PHENG, TEA LEE SENG, YANG XIAOYAN	}
\address{Institute of Systems Science, National University of Singapore, Singapore 119615}

\begin{document}
%\ninept
%
\maketitle
%
\begin{abstract}
% Briefly describe your problem statement and the proposed approach.

Singapore is a safe country. However, any gun shots become more dangerous in neighbourhood area, due to less awareness in public. It is critical to maintain safety here and automate alerting mechanism to Police will be time and life saving. 
We propose machine learning mechanism to understand background sound in Urban setup to identify gun shots. We apply techniques like 1D conv neural network, auto-encoder, LSTM, neural network on MFCC to understand which is good for recognizing gun shots in URBANSOUND8K DATASET.
 

\end{abstract}
%
\begin{keywords}
URBANSOUND8K DATASET, long short-term memory, auto encoder, mfcc, conv1D, MaxPooling1D, deep neural network, deep learning
\end{keywords}
%
\section{Introduction}
\label{sec:intro}
% Describe the problem you are working on and why it is important. 


Dangerous weapons can be a source of dangerous in a safe country.  Gun shots become more dangerous in neighbourhood area, due to less awareness in public. It is critical to maintain safety here and automate alerting mechanism to Police will be time and life saving. It is easpecially the case that police station may be far away from the incident area and they do not get alerted unless civilians reported it to them. Even with reports, the actual sound about the incident is lost and therefore losing original data.


% \begin{itemize}
%  \item The main report is provided in *.tex file
%  \item The reference is provided in *.bib file
%  \item The figures are provided as separate jpg/png files
% \end{itemize}


\section{Related work}

% Discuss published works or online references that relate to your project, such as \cite{adams1995hitchhiker}
% \cite{MikeSmalesSoundClassificationusingDeepLearning}

We download URBANSOUND8K DATASET from https://urbansounddataset.weebly.com/urbansound8k.html for analysis and model training.

We also refer to Ricky Kim's articles about UrbanSound Classification Part 1 \cite{RickyKimUrbanSoundClassificationPart1soundwavedigitalaudiosignal} and Part 2 \cite{RickyKimUrbanSoundClassificationPart2samplerateconversionLibrosa} for initial understanding of datasets. 

Mike's work \cite{MikeSmalesSoundClassificationusingDeepLearning} on 2d convolution deep neural network on MFCC also share us another direction on utilizing frequency domain representation for analysis. 


\section{Proposed approach}
\label{sec:proposed approach}
Based on the nature that all sounds wave files are 1 dimensional signals, sampling at different frequencies, we retrieved wave data via librosa library at default 22khz sampling rate.

We concatenates signals itself to 4 seconds long as signal feature.

We feed the signals to conv1d, auto-encoder, LSTM network for classification.

we also convert signals to mfcc format, apply mean on each mfcc stream and 3 layers dense neural network for classification.

Please refers experimental results section for details.

\section{Experimental results}
\label{sec:experimental results}

\begin{equation}\label{equation block model}
B_{r,c}=\sum\{f(i,j)|(i,j)\in \Omega_{r,c}\}.
\end{equation}
\begin{equation}\label{equation 1}
\sum_{x}=a+b+\hat{c},
\end{equation}

An inline equation is $a+b=c$. An example of two-column figure is provided in Figure \ref{figure1}, and the single-column figures is provided in Figure \ref{figure2}.

\begin{figure*}[tbh]\includegraphics[width=15cm]{iss.png}
    \caption{Test figure (two-column).\label{figure1}}
\end{figure*}


\begin{figure}[tbh]
    \centerline{\begin{tabular}{cc|c}
        \includegraphics[width=3cm]{iss.png}
        &\includegraphics[width=3cm]{iss.png}& text\\
    (a) & (b) & (c)
    \end{tabular}}
    \caption{Test figure (single column).\label{figure2}}
\end{figure}

\begin{table}[tbh]
\caption{The performance comparison.}\label{table1} \centerline{
    \begin{tabular}{clc|r}
    \hline\hline
    Approach & Ref. \cite{adams1995hitchhiker} & Ref. \cite{adams1995hitchhiker} & Proposed approach\\
    Metric A & $0.8181$ & $0.9171$ & $0.9616$ \\\hline
    Metric B & $0.8236$ & $0.7654$ & $0.8615$ \\\hline
    \end{tabular}
    }
\end{table}

\section{Conclusions}
\label{sec:conclusions}

Summarize your key results. What are limitations of your approach? Suggest ideas for future extensions of your ideas.


\bibliographystyle{IEEEbib}
\bibliography{references}

\end{document}
